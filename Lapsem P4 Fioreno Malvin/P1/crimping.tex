\section{Pendahuluan}
\subsection{Latar Belakang}
Firewall adalah komponen penting dalam keamanan dan manajemen jaringan modern. Firewall pertama kali dikembangkan pada akhir 1980-an sebagai solusi dari serangan cyber, seperti malware dan pembobolan jaringan. Fungsinya adalah untuk memantau dan mengontrol lalu lintas jaringan berdasarkan aturan yang telah ditentukan. NAT muncul pada awal 1990-an sebagai solusi untuk mengatasi keterbatasan alamat IPv4, sehingga banyak perangkat dalam suatu jaringan lokal bisa menggunakan satu public IP.

\subsection{Dasar Teori}
Firewall berfungsi sebagai penghalang antara jaringan internal dan eksternal, bekerja dengan mengfilter lalu lintas jaringan berdasarkan alamat IP, port, atau protokol. Firewall dapat berupa hardware, software, atau campuran dua duanya. NAT adalah teknik memodifikasi header paket jaringan untuk menerjemahkan alamat IP privat ke alamat IP publik atau sebaliknya, sehingga banyak perangkat bisa menggunakan satu alamat IP publik. NAT beroperasi pada router atau gateway.

%===========================================================%
\section{Tugas Pendahuluan}
Bagian ini berisi jawaban dari tugas pendahuluan yang telah anda kerjakan, beserta penjelasan dari jawaban tersebut
\begin{enumerate}
	\item Menggunkan port forwarding, di router aturlah NAT ke port80, protokol TCP dan destination internal 192.168.1.10
	\item Karena firewall berfungsi sebagai tameng utama dari serangan jaringan, NAT tidak mempunyai kemampuan tersebut
	\item Jaringan akan sangat rentan terhadap DDoS attack, dan penjebolan/serangan terhadap jaringan karena tidak ada proteksi.
\end{enumerate}