\section{Pendahuluan}
\subsection{Latar Belakang}
Dalam dunia jaringan komputer, komunikasi data yang handal dan efisien sangat bergantung pada dua aspek utama: infrastruktur fisik dan pengaturan logis jaringan. Infrastruktur fisik mencakup media transmisi seperti kabel dan konektor, sedangkan pengaturan logis mencakup pengalamatan IP dan mekanisme routing. Praktikum ini bertujuan untuk memberikan pemahaman langsung mengenai dua aspek penting tersebut, yaitu proses crimping kabel jaringan dan konfigurasi routing pada jaringan IPv4. Crimping diperlukan agar perangkat jaringan dapat terhubung dengan media transmisi yang sesuai, sedangkan routing digunakan untuk mengatur jalur komunikasi antar jaringan agar data dapat dikirim secara tepat dan efisien. Melalui kegiatan praktikum ini, mahasiswa diharapkan dapat memahami dan mempraktikkan bagaimana membangun jaringan dari sisi fisik maupun logis, serta mampu mengonfigurasi jaringan IPv4 agar mendukung komunikasi antar perangkat dengan benar.

\subsection{Dasar Teori}
Dalam membangun jaringan komputer, terdapat dua aspek yang sangat penting, yaitu aspek fisik dan aspek logis. Salah satu aspek fisik yang mendasar adalah proses crimping, yaitu teknik menyambungkan konektor RJ-45 ke ujung kabel UTP (Unshielded Twisted Pair) dengan menggunakan alat khusus yang disebut crimping tool. Kabel UTP merupakan salah satu media transmisi yang umum digunakan untuk menghubungkan perangkat jaringan seperti komputer, switch, dan router. Terdapat dua jenis konfigurasi kabel yang biasa digunakan, yaitu straight-through yang digunakan untuk menghubungkan perangkat berbeda jenis (misalnya komputer ke switch), dan crossover yang digunakan untuk perangkat sejenis (misalnya komputer ke komputer). Sementara itu, aspek logis dalam jaringan komputer erat kaitannya dengan pengalamatan dan pengaturan jalur komunikasi. IPv4 (Internet Protocol version 4) adalah sistem pengalamatan yang digunakan untuk mengidentifikasi setiap perangkat dalam jaringan. Alamat IPv4 terdiri dari 32 bit yang dibagi menjadi empat oktet dan ditulis dalam format desimal bertitik, seperti 192.168.1.1. Dalam implementasinya, pengaturan alamat IP juga sering dikombinasikan dengan subnetting, yaitu proses membagi suatu jaringan besar menjadi beberapa jaringan kecil (subnet) agar penggunaan IP lebih efisien dan manajemen jaringan menjadi lebih mudah. Subnetting dilakukan dengan menentukan subnet mask, yang memisahkan bagian alamat IP yang menunjukkan jaringan dan bagian yang menunjukkan host. Untuk memungkinkan komunikasi antar jaringan yang berbeda, diperlukan proses routing, yaitu penentuan jalur terbaik bagi paket data untuk mencapai tujuannya. Routing dapat dilakukan secara statis (static routing), di mana administrator menentukan jalur secara manual, atau secara dinamis menggunakan protokol routing. Dalam konteks praktikum ini, konfigurasi routing dilakukan pada jaringan berbasis IPv4 untuk memastikan bahwa setiap paket data yang dikirim dari satu subnet dapat sampai ke subnet lain dengan benar melalui perangkat router.

%===========================================================%
\section{Tugas Pendahuluan}
\subsection{Alokasi IP Address dan Prefix (CIDR)}

Perencanaan IP address dilakukan menggunakan alamat private IPv4 (\texttt{192.168.0.0/24}) dan disesuaikan dengan kebutuhan perangkat di setiap departemen. Alokasi IP dilakukan secara efisien menggunakan teknik subnetting agar tidak terjadi pemborosan.

\begin{table}[H]
\centering
\caption{Perencanaan Subnet untuk Tiap Departemen}
\begin{tabular}{|l|c|c|c|c|l|}
\hline
\textbf{Departemen} & \textbf{Perangkat} & \textbf{CIDR} & \textbf{IP/Subnet} & \textbf{Total IP} & \textbf{Rentang IP} \\
\hline
R\&D & 100 & /25 & 192.168.0.0/25 & 128 & 192.168.0.1 -- 192.168.0.126 \\
Produksi & 50 & /26 & 192.168.0.128/26 & 64 & 192.168.0.129 -- 192.168.0.190 \\
Administrasi & 20 & /27 & 192.168.0.192/27 & 32 & 192.168.0.193 -- 192.168.0.222 \\
Keuangan & 10 & /28 & 192.168.0.224/28 & 16 & 192.168.0.225 -- 192.168.0.238 \\
\hline
\end{tabular}
\end{table}

\subsection{Topologi Jaringan Sederhana}

Topologi jaringan perusahaan menggunakan satu router utama yang menghubungkan empat subnet dari tiap departemen. Berikut adalah gambaran topologi secara sederhana:

\begin{center}
\begin{tikzpicture}[node distance=2cm, every node/.style={draw, rectangle, minimum width=2.5cm, minimum height=1cm}]
\node (router) {Router};
\node (rnd) [below left=of router] {LAN R\&D};
\node (prod) [below=of router] {LAN Produksi};
\node (adm) [below right=of router] {LAN Administrasi};
\node (keu) [right=4cm of adm] {LAN Keuangan};

\draw[-] (router) -- (rnd);
\draw[-] (router) -- (prod);
\draw[-] (router) -- (adm);
\draw[-] (router.east) -- ++(1,0) |- (keu.north);
\end{tikzpicture}
\end{center}


\subsection{Tabel Routing}

Tabel routing berikut digunakan pada router untuk menghubungkan seluruh subnet:

\begin{table}[H]
\centering
\caption{Tabel Routing Sederhana}
\begin{tabular}{|l|c|l|c|}
\hline
\textbf{Network Destination} & \textbf{Prefix} & \textbf{Gateway} & \textbf{Interface} \\
\hline
192.168.0.0 & /25 & - & eth0 (R\&D) \\
192.168.0.128 & /26 & - & eth1 (Produksi) \\
192.168.0.192 & /27 & - & eth2 (Administrasi) \\
192.168.0.224 & /28 & - & eth3 (Keuangan) \\
\hline
\end{tabular}
\end{table}

\subsection{Jenis Routing yang Cocok dan Justifikasi}
Jenis routing yang paling sesuai untuk perusahaan ini adalah \textbf{Static Routing}. Hal ini disebabkan karena jumlah subnet yang digunakan relatif sedikit dan topologi jaringan bersifat tetap atau jarang mengalami perubahan. Dengan menggunakan static routing, administrator dapat mengatur rute secara manual untuk masing-masing jaringan tanpa memerlukan protokol dinamis tambahan. Keuntungan lain dari static routing adalah mudah dikonfigurasi, hemat sumber daya perangkat, dan cocok untuk jaringan berskala kecil hingga menengah. Namun, jika di masa depan perusahaan mengalami ekspansi jaringan yang lebih kompleks, maka penggunaan \textbf{Dynamic Routing} seperti \textit{OSPF} dapat dipertimbangkan.
