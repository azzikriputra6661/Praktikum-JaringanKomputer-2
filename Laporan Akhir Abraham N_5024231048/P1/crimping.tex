% This file should be saved as crimping.tex in the P1 folder
% as referenced in the main document's \section{Pendahuluan}
\subsection*{Latar Belakang}

Seiring dengan pesatnya pertumbuhan perangkat yang terhubung ke internet, kebutuhan akan alamat IP yang lebih luas menjadi semakin mendesak. Protokol IPv4 yang telah lama digunakan memiliki keterbatasan dalam jumlah alamat, hanya menyediakan sekitar 4,3 miliar alamat yang sebagian besar telah terpakai. Untuk mengatasi permasalahan tersebut, dikembangkan protokol baru yaitu IPv6 (Internet Protocol version 6), yang menyediakan ruang alamat yang jauh lebih besar serta fitur-fitur modern yang tidak tersedia pada IPv4.

Dalam implementasinya, penggunaan IPv6 tidak hanya sebatas pada pemberian alamat IP, namun juga mencakup manajemen dan pengaturan routing agar komunikasi antar jaringan berjalan dengan baik. Routing pada IPv6 memiliki prinsip yang mirip dengan IPv4, namun dengan penyesuaian pada format alamat dan mekanisme tertentu. Oleh karena itu, pemahaman mengenai konfigurasi alamat IPv6 serta pengelolaan routing, baik secara statis maupun dinamis, sangat penting untuk mendukung infrastruktur jaringan yang andal dan efisien.

Praktikum ini bertujuan untuk mengenalkan dan melatih mahasiswa dalam melakukan konfigurasi dan manajemen jaringan berbasis IPv6, mulai dari pemberian alamat, pengujian koneksi, hingga penerapan routing statis maupun dinamis antar subnet. Melalui praktik langsung, diharapkan mahasiswa dapat memahami konsep dasar dan langkah-langkah teknis yang dibutuhkan dalam implementasi jaringan IPv6.

\subsection*{Dasar Teori}

\textbf{Internet Protocol Version 6 (IPv6)} merupakan generasi terbaru dari protokol internet yang dirancang untuk menggantikan IPv4. IPv6 menggunakan alamat sepanjang 128 bit, yang memungkinkan pengalamatan hingga $2^{128}$ alamat unik, jauh lebih banyak dibandingkan dengan 32 bit pada IPv4. Alamat IPv6 ditulis dalam format heksadesimal yang dipisahkan dengan tanda titik dua (:), contohnya: \texttt{2001:db8::1}.

Fitur-fitur penting pada IPv6 antara lain:
\begin{itemize}
    \item \textbf{Ruang alamat lebih besar}: Memungkinkan lebih banyak perangkat terhubung.
    \item \textbf{Header lebih sederhana}: Mempercepat pemrosesan paket oleh router.
    \item \textbf{Autokonfigurasi}: Mendukung Stateless Address Autoconfiguration (SLAAC).
    \item \textbf{Keamanan bawaan}: Mendukung IPSec sebagai standar protokol keamanan.
\end{itemize}

\textbf{Routing} adalah proses pengiriman paket data dari satu jaringan ke jaringan lain menggunakan router. Pada IPv6, terdapat dua jenis routing yang umum digunakan:
\begin{enumerate}
    \item \textbf{Routing Statis}: Konfigurasi rute dilakukan secara manual oleh administrator. Cocok digunakan untuk jaringan kecil atau dengan topologi yang tidak sering berubah.
    \item \textbf{Routing Dinamis}: Menggunakan protokol seperti RIPng atau OSPFv3 untuk membangun dan memperbarui tabel rute secara otomatis berdasarkan kondisi jaringan.
\end{enumerate}

\textbf{Manajemen IPv6} mencakup kegiatan seperti pengalokasian alamat IP untuk antarmuka jaringan, pengecekan konektivitas, konfigurasi gateway, dan pengaturan rute agar jaringan dapat saling terhubung. Pemahaman terhadap struktur subnet IPv6 serta cara membagi dan mengalokasikan alamat sangat penting dalam proses ini.

Dengan menguasai dasar teori ini, mahasiswa diharapkan mampu menerapkan konfigurasi jaringan IPv6 secara efektif dalam lingkungan praktikum maupun dunia nyata.


%===========================================================%
\section*{Tugas Pendahuluan}

\subsection*{1. Jelaskan apa itu IPv6 dan apa bedanya dengan IPv4}

IPv6 (Internet Protocol version 6) merupakan versi terbaru dari protokol internet yang dikembangkan untuk menggantikan IPv4. Perbedaan utama antara keduanya terletak pada panjang alamat: IPv4 memiliki panjang 32 bit, sedangkan IPv6 memiliki panjang 128 bit. Dengan panjang alamat yang jauh lebih besar, IPv6 dapat menyediakan jumlah alamat IP yang sangat banyak, yaitu sekitar $3.4 \times 10^{38}$ alamat, dibandingkan dengan IPv4 yang hanya menyediakan sekitar 4,3 miliar alamat. Selain itu, IPv6 mendukung fitur-fitur baru seperti autokonfigurasi alamat (SLAAC), keamanan bawaan melalui IPSec, dan struktur header yang lebih efisien. IPv6 ditulis dalam notasi heksadesimal yang dipisahkan oleh tanda titik dua (:), sedangkan IPv4 menggunakan notasi desimal dengan pemisah titik (.).

\subsection*{2. Sebuah organisasi mendapatkan blok alamat IPv6 2001:db8::/32}

\textbf{a.} Blok alamat IPv6 \texttt{2001:db8::/32} perlu dibagi menjadi empat subnet dengan prefix /64. Karena setiap subnet IPv6 biasanya menggunakan prefix /64, maka pembagian dapat dilakukan dengan menambahkan 32 bit ke dalam struktur alamat. Empat subnet yang dihasilkan adalah sebagai berikut:
\begin{itemize}
    \item Subnet A: \texttt{2001:db8:0:1::/64}
    \item Subnet B: \texttt{2001:db8:0:2::/64}
    \item Subnet C: \texttt{2001:db8:0:3::/64}
    \item Subnet D: \texttt{2001:db8:0:4::/64}
\end{itemize}

\textbf{b.} Hasil alokasi alamat IPv6 untuk masing-masing subnet adalah:
\begin{itemize}
    \item Subnet A: \texttt{2001:db8:0:1::/64}
    \item Subnet B: \texttt{2001:db8:0:2::/64}
    \item Subnet C: \texttt{2001:db8:0:3::/64}
    \item Subnet D: \texttt{2001:db8:0:4::/64}
\end{itemize}

\subsection*{3. Asumsikan terdapat sebuah router yang menghubungkan keempat subnet melalui empat antarmuka}

\textbf{a.} Setiap antarmuka pada router akan dihubungkan ke masing-masing subnet dan diberi alamat IPv6 sebagai berikut:
\begin{itemize}
    \item ether1 (Subnet A): \texttt{2001:db8:0:1::1/64}
    \item ether2 (Subnet B): \texttt{2001:db8:0:2::1/64}
    \item ether3 (Subnet C): \texttt{2001:db8:0:3::1/64}
    \item ether4 (Subnet D): \texttt{2001:db8:0:4::1/64}
\end{itemize}

\textbf{b.} Konfigurasi IP address IPv6 pada masing-masing antarmuka router (misalnya pada router MikroTik) dapat dilakukan dengan perintah berikut:

\begin{verbatim}
/interface ethernet
set [ find default-name=ether1 ] name=ether1-subnetA
set [ find default-name=ether2 ] name=ether2-subnetB
set [ find default-name=ether3 ] name=ether3-subnetC
set [ find default-name=ether4 ] name=ether4-subnetD

/ipv6 address
add address=2001:db8:0:1::1/64 interface=ether1-subnetA
add address=2001:db8:0:2::1/64 interface=ether2-subnetB
add address=2001:db8:0:3::1/64 interface=ether3-subnetC
add address=2001:db8:0:4::1/64 interface=ether4-subnetD
\end{verbatim}

\subsection*{4. Daftar IP Table berupa daftar rute statis agar semua subnet dapat saling berkomunikasi}

Karena keempat subnet dihubungkan melalui satu router pusat, maka komunikasi antar subnet dapat dilakukan langsung. Namun, jika terdapat router tambahan yang ingin mengakses subnet tersebut, maka diperlukan penambahan rute statis. Contoh konfigurasi rute statis untuk router kedua yang terhubung ke router utama melalui ether1 adalah sebagai berikut:

\begin{verbatim}
/ipv6 route
add dst-address=2001:db8:0:1::/64 gateway=fe80::1%ether1
add dst-address=2001:db8:0:2::/64 gateway=fe80::1%ether1
add dst-address=2001:db8:0:3::/64 gateway=fe80::1%ether1
add dst-address=2001:db8:0:4::/64 gateway=fe80::1%ether1
\end{verbatim}

\subsection*{5. Fungsi routing statis pada jaringan IPv6 dan kapan digunakan}

Routing statis pada jaringan IPv6 berfungsi untuk mengatur jalur lalu lintas paket secara manual antar subnet atau antar router. Dengan routing statis, administrator dapat menentukan rute tertentu tanpa bergantung pada protokol dinamis. Routing statis cocok digunakan pada jaringan kecil, topologi sederhana, atau ketika konfigurasi tidak sering berubah. Keuntungannya adalah lebih mudah dikontrol dan dipantau, serta tidak memerlukan banyak sumber daya. Namun, pada jaringan yang besar dan kompleks, routing dinamis seperti OSPFv3 atau RIPng lebih efisien karena dapat menyesuaikan dengan perubahan kondisi jaringan secara otomatis.





\section{Langkah-Langkah Percobaan}
Dalam percobaan crimping dan routing IPv4, kami memulai dengan mempersiapkan peralatan yang dibutuhkan, seperti kabel UTP, konektor RJ45, tang crimping, dan LAN tester. Kami menyusun kabel UTP dengan mengikuti urutan warna yang benar sesuai standar T568B untuk memastikan koneksi yang tepat. 

\begin{figure}[H]
    \centering
    \includegraphics[width=0.48\textwidth]{P1/img/Crimping 1.jpeg}
    \caption{Proses penyusunan kabel}
    \label{fig:crimping1}
\end{figure}

Setelah itu, kami menggunakan tang crimping untuk memasang konektor RJ45 pada ujung kabel. 

\begin{figure}[H]
    \centering
    \includegraphics[width=0.48\textwidth]{P1/img/Crimping 2.jpeg}
    \caption{Kabel sudah tersusun sebelum crimping}
    \label{fig:crimping2}
\end{figure}

Untuk memverifikasi hasil crimping, kami menguji kabel dengan LAN tester untuk memastikan bahwa setiap kabel terhubung dengan benar. 

\begin{figure}[H]
    \centering
    \includegraphics[width=0.48\textwidth]{P1/img/Crimping 3.jpeg}
    \caption{Percobaan kabel dengan LAN Tester}
    \label{fig:crimping3}
\end{figure}

Kabel hasil crimping yang telah selesai 

\begin{figure}[H]
    \centering
    \includegraphics[width=0.48\textwidth]{P1/img/Crimping 4.jpeg}
    \caption{Foto Kabel}
    \label{fig:crimping4}
\end{figure}

\newpage
Pada percobaan Routing IPv4 menggunakan aplikasi Winbox, saya memulai dengan mereset kedua router ke kondisi awal untuk menghindari konflik konfigurasi. Langkah pertama adalah membuka aplikasi Winbox dan login ke kedua router menggunakan MAC address atau IP default, dengan username "admin" dan tanpa password jika belum diatur. Setelah login berhasil, saya mengonfigurasi IP address pada interface ether1 yang digunakan untuk menghubungkan kedua router. Pada router pertama, saya memberikan IP address 10.10.10.1/30, dan pada router kedua saya memberikan IP address 10.10.10.2/30. Setelah itu, saya melanjutkan dengan mengonfigurasi interface ether2 yang akan digunakan untuk menghubungkan router dengan jaringan LAN. Pada router pertama, saya memberikan IP address 192.168.10.1/27, sementara pada router kedua saya memberikan IP address 192.168.20.1/27 untuk jaringan LAN masing-masing. 

Selanjutnya, saya menambahkan routing statis untuk memastikan paket data dapat diteruskan antar router. Saya membuka menu IP -> Routes di Winbox, kemudian menambahkan rute untuk masing-masing router. Pada router pertama, saya menambahkan rute dengan Dst. Address 192.168.20.0/27 dan gateway 10.10.10.2, sementara pada router kedua saya menambahkan rute dengan Dst. Address 192.168.10.0/27 dan gateway 10.10.10.1. Setelah konfigurasi routing selesai, saya mengonfigurasi IP address secara manual pada laptop yang terhubung ke kedua router. Laptop yang terhubung ke router pertama saya berikan IP address 192.168.10.2 dengan gateway 192.168.10.1, sedangkan laptop yang terhubung ke router kedua saya berikan IP address 192.168.20.2 dengan gateway 192.168.20.1. 

Terakhir, saya melakukan ping untuk menguji konektivitas dari laptop yang terhubung ke router pertama ke laptop yang terhubung ke router kedua. Jika ping berhasil, itu berarti konfigurasi routing IPv4 telah berhasil dan komunikasi antar router berjalan dengan baik.

\newpage
\subsection*{Percobaan Statis (Gagal)}
\begin{figure}[H]
  \centering
  \begin{minipage}[t]{0.48\textwidth}
    \centering
    \includegraphics[width=\linewidth]{P1/img/PC1->Router1.jpeg}
    \caption{PC 1 -> Router 1}
    \label{fig:pc1router1}
  \end{minipage}
  \hfill
  \begin{minipage}[t]{0.48\textwidth}
    \centering
    \includegraphics[width=\linewidth]{P1/img/Router1->Router2.jpeg}
    \caption{Router 1 -> Router 2}
    \label{fig:router1router2}
  \end{minipage}
\end{figure}

\begin{figure}[H]
  \centering
  \begin{minipage}[t]{0.48\textwidth}
    \centering
    \includegraphics[width=\linewidth]{P1/img/Router2->Router1.jpeg}
    \caption{Router 2 -> Router 1}
    \label{fig:router2router1}
  \end{minipage}
  \hfill
  \begin{minipage}[t]{0.48\textwidth}
    \centering
    \includegraphics[width=\linewidth]{P1/img/PC2->Router2.jpeg}
    \caption{PC 2 -> Router 2}
    \label{fig:pc2router2}
  \end{minipage}
  \caption*{Proses percobaan Routing IPv4}
\end{figure}

\newpage
\section{Analisis Hasil Percobaan}

Pada percobaan Crimping, kami berhasil melaksanakan proses crimping kabel dengan menggunakan alat yang telah disiapkan, namun pada kabel LAN pertama, kami menemui kendala karena urutan kabel yang tidak sesuai standar, yang menyebabkan koneksi tidak terhubung dengan baik. Hal ini mengharuskan kami untuk mengulang proses crimping, dengan memotong kabel tersebut dan menyusunnya kembali sesuai urutan warna yang benar. Setelah melakukan pengulangan pada percobaan kedua, kabel berhasil terpasang dengan benar dan koneksi pun berjalan lancar.

Untuk percobaan Routing IPv4, kami mengalami kendala dalam melaksanakan konfigurasi IPv4 statis. Meskipun kami telah mengikuti prosedur yang benar, kami tidak berhasil menyelesaikan percobaan ini hingga waktu habis. Masalah utama yang kami temui adalah kesalahan pada pengaturan alamat IP, di mana pada laptop kedua yang terhubung dengan router kedua, terdapat alamat IP bawaan dari Google yang tidak dapat dihapuskan, yang menyebabkan koneksi terus terinterupsi. Hal ini menghambat komunikasi antar perangkat, sehingga konfigurasi statis tidak dapat berhasil. Selain itu, kami juga mencoba untuk melaksanakan routing IPv4 dinamis, namun karena waktu yang terbatas, kami tidak dapat menyelesaikan percobaan ini. Dengan waktu yang hampir habis, kami tidak dapat mengonfigurasi routing dinamis menggunakan protokol RIP yang direncanakan.

\newpage
\section{Hasil Tugas Modul}
Dalam praktikum ini, saya membangun jaringan internal untuk sebuah perusahaan dengan membagi jaringan berdasarkan departemen-departemen yang ada. Setiap departemen memiliki jaringan lokalnya sendiri yang terhubung melalui satu router utama. Untuk mempermudah pengelolaan, saya memutuskan untuk menghubungkan satu komputer yang mewakili sepuluh perangkat end user, karena menghubungkan lebih dari seratus perangkat akan menghabiskan waktu dan sumber daya.

Proses dimulai dengan menyiapkan sebuah router yang menghubungkan dua switch utama. Switch di sisi kiri router saya hubungkan dengan departemen Produksi dan Administrasi, sedangkan switch di sisi kanan menghubungkan departemen Keuangan dan R\&D. Router saya konfigurasi dengan menggunakan alamat IP 192.168.1.1 untuk interface g0/0 yang terhubung ke switch kiri, dan alamat IP 192.168.2.1 untuk interface g0/1 yang terhubung ke switch kanan. Setelah itu, saya melakukan alokasi IP untuk masing-masing departemen:
untuk Produksi menggunakan IP 192.168.1.51 hingga 192.168.1.100, Administrasi menggunakan 192.168.1.21 hingga 192.168.1.40, Keuangan menggunakan 192.168.2.11 hingga 192.168.2.20, dan R\&D menggunakan 192.168.2.101 hingga 192.168.2.200.

Selanjutnya, saya melakukan percobaan pengiriman pesan antar end user untuk menguji konektivitas antar departemen. Saya mencoba mengirim pesan dari PC di departemen Produksi ke departemen Keuangan dan R\&D, serta sebaliknya dari PC Keuangan dan R\&D ke departemen Administrasi. Hasilnya, pesan berhasil dikirim dan diterima dengan sukses, yang menunjukkan bahwa konfigurasi jaringan dan routing yang saya buat berfungsi dengan baik.

\begin{figure}[H]
  \centering
  \begin{minipage}[t]{0.48\textwidth}
    \centering
    \includegraphics[width=\linewidth]{P1/img/Struktur.jpeg}
    \caption{Struktur Jaringan Keseluruhan}
    \label{fig:struktur}
  \end{minipage}
  \hfill
  \begin{minipage}[t]{0.48\textwidth}
    \centering
    \includegraphics[width=\linewidth]{P1/img/CLI Config.jpeg}
    \caption{CLI Router Utama}
    \label{fig:cli}
  \end{minipage}
\end{figure}

\newpage
\section{Struktur Jaringan}

\begin{figure}[H]
  \centering
  \begin{minipage}[t]{0.48\textwidth}
    \centering
    \includegraphics[width=\linewidth]{P1/img/R&DIP.jpeg}
    \caption{Address R\&D}
    \label{fig:rnd}
  \end{minipage}
  \hfill
  \begin{minipage}[t]{0.48\textwidth}
    \centering
    \includegraphics[width=\linewidth]{P1/img/FinIP.jpeg}
    \caption{Address Finance}
    \label{fig:finance}
  \end{minipage}
\end{figure}

\newpage
\section{Jaringan Bagian Kanan}

\begin{figure}[H]
  \centering
  \begin{minipage}[t]{0.48\textwidth}
    \centering
    \includegraphics[width=\linewidth]{P1/img/ProdIP.jpeg}
    \caption{Address Production}
    \label{fig:production}
  \end{minipage}
  \hfill
  \begin{minipage}[t]{0.48\textwidth}
    \centering
    \includegraphics[width=\linewidth]{P1/img/AdminIP.jpeg}
    \caption{Address Administration}
    \label{fig:admin}
  \end{minipage}
\end{figure}

\begin{figure}[H]
  \centering
  \includegraphics[width=0.48\textwidth]{P1/img/Connected.jpeg}
  \caption{Seluruh Jaringan terkoneksi}
  \label{fig:connected}
\end{figure}

\newpage
\section{Lampiran}
\subsection{Dokumentasi saat praktikum}

\begin{figure}[H]
    \centering
    \includegraphics[width=0.48\textwidth]{P1/img/Crimping 2.jpeg}
    \caption{Kabel sudah tersusun sebelum crimping}
    \label{fig:crimping2_lampiran}
\end{figure}

Untuk memverifikasi hasil crimping, kami menguji kabel dengan LAN tester untuk memastikan bahwa setiap kabel terhubung dengan benar. 

\begin{figure}[H]
    \centering
    \includegraphics[width=0.48\textwidth]{P1/img/Crimping 3.jpeg}
    \caption{Percobaan kabel dengan LAN Tester}
    \label{fig:crimping3_lampiran}
\end{figure}

Kabel hasil crimping yang telah selesai 

\begin{figure}[H]
    \centering
    \includegraphics[width=0.48\textwidth]{P1/img/Crimping 4.jpeg}
    \caption{Foto Kabel}
    \label{fig:crimping4_lampiran}
\end{figure}


