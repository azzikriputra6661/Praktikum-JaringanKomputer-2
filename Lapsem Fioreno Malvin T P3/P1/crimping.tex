\section{Pendahuluan}
\subsection{Latar Belakang}
 Pengembangan jaringan nirkabel didorong oleh kebutuhan akan mobilitas dan fleksibilitas dalam berkomunikasi dan mengakses informasi. Sejarahnya dimulai pada akhir abad ke-19 dengan penemuan gelombang radio oleh Heinrich Hertz. Tujuan utamanya adalah mengatasi keterbatasan kabel fisik, memungkinkan jaringan bisa terhubung kapan saja, dimana saja.

\subsection{Dasar Teori}
 Data digital dari perangkat diubah menjadi sinyal radio yang kemudian dipancarkan melalui antena. Sinyal ini merambat di udara dan diterima oleh antena perangkat lain, yang kemudian mengubahnya kembali menjadi data digital. Ada juga rentang frekuensi yang dialokasikan untuk berbagai jenis sinyal wireless agar tidak saling mengganggu satu sama lain, meminimalisir noise dan data loss.

%===========================================================%
\section{Tugas Pendahuluan}
Bagian ini berisi jawaban dari tugas pendahuluan yang telah anda kerjakan, beserta penjelasan dari jawaban tersebut
\begin{enumerate}
	\item Tergantung situasi dan penggunaannya. Bila ingin meminimalisir noise dan error serta latency, wired lah jawabannya. Namun bila ingin portabilitas dan kemudahan penggunaan, wireless lah yang harus digunakan.
	\item Router bertugas meng-assign IP dan routing traffic pada jaringan kepada perangkat - perangkat yang terhubung. Access Point berfungsi menmperluas atau membuat jaringan wireless (WLAN). Dan modem berfungsi mengubah sinyal digital dari perangkat terhubung menjadi sinyal analog untuk dikirim misalnya ke provider internet melalui kabel fiber optik, serta sebaliknya.
	\item Wireless point to point bridge, perangkat ini mampu mentransmisikan jaringan wireless dalam jarak jauh dengan efisiensi yang sangat tinggi.
\end{enumerate}